\documentclass[english,a4paper]{article}
\usepackage[utf8]{inputenc}
\usepackage{babel,moreverb}
\usepackage[pdftex]{graphicx}
\usepackage{wrapfig}
\usepackage{float}
\usepackage{listings}
\usepackage{color}
 
\lstset{ %
language=Perl,                  % choose the language of the code
basicstyle=\small,              % the size of the fonts that are used for the code
numbers=none,                   % where to put the line-numbers
numberstyle=\footnotesize,      % the size of the fonts that are used for the line-numbers
stepnumber=2,                   % the step between two line-numbers. If it's 1 each line will be numbered
numbersep=5pt,                  % how far the line-numbers are from the code
backgroundcolor=\color{white},  % choose the background color. You must add \usepackage{color}
showspaces=false,               % show spaces adding particular underscores
showstringspaces=false,         % underline spaces within strings
showtabs=false,                 % show tabs within strings adding particular underscores
frame=single,                   % adds a frame around the code
tabsize=2,                      % sets default tabsize to 2 spaces
captionpos=b,                   % sets the caption-position to bottom
breaklines=true,                % sets automatic line breaking
breakatwhitespace=false,        % sets if automatic breaks should only happen at whitespace
escapeinside={\%*}{*)}          % if you want to add a comment within
                                % your code
}

\newcommand{\HRule}{\rule{\linewidth}{0.5mm}}

\begin{document}

\begin{titlepage}
 
\begin{center}
 
 
% Upper part of the page
%\includegraphics[width=0.15\textwidth]{./logo}\\[1cm]
 
% \textsc{\LARGE Yggdrasil documentation}\\[1.5cm]

\textsc{Yggdrasil documentation}\\[1.5cm]

 
\textsc{\Large Temporal storage of simple objects}\\[0.5cm]
 
 
% Title
\HRule \\[0.5cm]
{ \huge \bfseries Overview of Yggdrasil}\\[0.4cm]
 
\HRule \\[1.5cm]
 
%\begin{minipage}{0.4\textwidth}
%\begin{flushleft} \large
%\emph{Author:}\\
%Terje \textsc{Kvernes}
%\end{flushleft}
%\end{minipage}
%\begin{minipage}{0.4\textwidth}
%\begin{flushright} \large
%\emph{Author:} \\
%David \textsc{Ranvig}
%\end{flushright}
%\end{minipage}

Terje Kvernes \& David Ranvig

\vfill
 
% Bottom of the page
{\large \today}
 
\end{center}
 
\end{titlepage}


\tableofcontents

\newpage

\section{Introductions}

\subsection{What is Yggdrasil?}

Yggdrasil aims to a ``dynamic relational temporal object database''.
In essence, Yggdrasil aims to add two abstractions to the traditional
view of a relational database: implicit temporal storage and a simple
object model to represent the data stored.  In addition to this
Yggdrasil allows the relations of the entities stored within to be
altered, and new entities and their relations to be added while the
system is running.  The relations are described by the administrator
of the system and as soon as any relation is described to Yggdrasil,
it is added to the overall structure of the installation.  

In Yggdrasil lingo you can think of an entity as a ``class'' from the
OO world.  The ``object'' is an instance of this entity, each object
existing in several temporal versions within the system.

Initializing Yggdrasil has one special and specific feature.  You need
to give Yggdrasil a namespace to work within.  This ``namespace''
parameter defines which namespace will house the entities you access
This is, in essence, your class hierarchy.  You'll want to ensure it's
uncluttered.  The rest of the parameters are all sent to the back end
storage layer, and both their meaning and their necessity varies
depending on said layer.  Look at the documentation for
Yggdrasil::Storage and its engines if in doubt.

\subsection{Objects}

An object is an instance of an entity within the system.  This
instance is the primary working set that Yggdrasil operates on.
Objects contain properties which are key / value pairs.  Objects are
identified by a unique name within each entity.

%\begin{wrapfigure}{l}{5cm}
%  \begin{center}
%    \includegraphics[scale=0.1]{/usr/share/openclipart/png/computer/icons/lemon-theme/actions/messagebox_warning}
%  \end{center}
%  \small{You might see ``objects'' referred to as ``entities'' in Yggdrasil}
%\end{wrapfigure}

An object within Yggdrasil isn't a singular instance of grouped data.
As any change to this object is kept, every version of the object
throughout its existence is stored.  The default object is the
``current'' object, defined as the set of properties that are
currently active and not expired.

The limitations are currently bound to the objects being fairly
simple, they are not allowed to store anything more complex than
anything that can be mapped into a Yggdrasil type\footnote{See table
  \ref{types} on page \pageref{types} for more information on
  Yggdrasil types}, and their only relations to other objects are the
ones defined by linking objects together.  References will be
flattened, so you don't want to store them in Yggdrasil.  Using
Storable is usually a sign something is wrong, either due to something
missing from Yggdrasil or just faulty usage of the library.


\subsubsection{Properties}

Depending on the way Yggdrasil is set up you may or may note have
defined types for your properties, and you may or may not have
constraints to the data stored for each property.  

Yggdrasil can either be flexible and treat all property values as a
default type, or you can select between a set of types Yggdrasil
guarantees no matter the back end it's running on.  The supported
types are listed in the type table on page \pageref{types}.
 
\subsection{Inheritance} 

Yggdrasil supports inheritance between entities.  This inheritance
allows relations and properties of a parent class to be accessible to
any of its children, at any depth.

\subsection{Relations}

To prepare Yggdrasil for relations between objects, a relation between
the entities the objects represent needs to be defined.  This is to
say one defines the relations between entities, then objects are
linked together on an object level as desired.

This link allows Yggdrasil to trace connections within the system, and
Yggdrasil will follow any links to any length it needs to find all
possible relations between objects.  The idea here is that the user is
freed from providing anything but direct relations, while Yggdrasil
will do the work of building a network from this information.

\subsection{Temporal dimension}

A principal idea of Yggdrasil is that data is only inserted, never
deleted.  This also means that there is a clear distinction between
``available'' and ``current'' data contained within the system.
As long as no time frame, or slice, is requested, all requests work on
the ``current'' dataset. 

% Deletion only happens if and only if the object structure (``entity''
% in Yggdrasil lingo) they belong to is deleted, and that deleted
% structure is ``purged''.  Deletion of an entity is therefore, in some
% circumstances, reversible.

Changes in the structure will retain the information if possible, the
system will inform you at any time if any action you take will
permanently delete any data.

\subsection{Dynamic storage}

New entities and new relations between entities can be issued on the
fly while the system is running live.  

\subsection{Moving and renaming structure}

Yggdrasil supports moving and renaming entities and properties.
Moving an entity changes the entities place in the inheritance
hierarchy to becoming a descendant of the target entity.  Renaming an
entity simply changes the entities name.  Moving a property moves the
property from the one entity to another, whilst renaming does the
obvious thing.

Moving and renaming structure is a way to dynamically allow for
entities and properties to exist in two or more places at the same
time within Yggdrasil.  The objective is to allow an administrator to
provide backwards compatibility on a running system while at the same
time performing the desired structural changes.  As long as a move or
rename is active, Yggdrasil will refuse any structural changes that
would collide with the namespaces under the effects of the move or
rename.  In essence, any namespace that is active, moved or renamed is
considered ``locked'' and unavailable.

Once a move or rename is considered obsolete, that is, the
compatibility layer provided is no longer required, the move or rename
can be expired.  This frees up the namespace that was in use.  It
might be prudent to leave this namespace open for a while if you're
unsure if every Yggdrasil client is updated to use the new structure.

Neither operation is allowed if a namespace collision would occur, and
as described above, move and rename will also fail if there is a
currently active \textbf{move} or \textbf{rename} at the destination.

\newpage
\section{Using Yggdrasil}

\subsection{The basics}

Yggdrasil has a few design rules it tries very hard to stay true to.
One of the big ones here is the only objects you'll ever see are those
from your namespace.  Neither Yggdrasil nor its own classes ever
return objects to user space.  This has been done to avoid any
thoughts about Yggdrasil meta structures having an impact on the data
that is being stored.  Your objects are your data, always.

\subsection{Initializing Yggdrasil}

Initializing Yggdrasil is fairly straight forward.

\lstset{caption=Initializing Yggrasil,label=ygginit,float=tp,aboveskip=0.7cm}
\begin{lstlisting}
new Yggdrasil(
              user      => user,
              password  => password,
              host      => host,
              db        => databasename,
              engine    => engine,
              namespace => 'Ygg',
              mapper    => 'SHA1',
             );
\end{lstlisting}

\subsection{Creating entities}

When an entity, say ``Host'', is defined within Yggdrasil, access will
be created to the class ``Ygg::Host''.  as we earlier defined the
namespace for Yggdrasil to work in previously to be named ``Ygg''.
We'll then create a ``Host'' object called ``nommo'' and one called
``ninhursaga''.

\lstset{caption=Defining entities,label=entity}
\begin{lstlisting}
$hostclass = define Yggdrasil::Entity 'Host';
my $nommo = $hostclass->new( 'nommo' );
my $ninhursaga = Ygg::Host->new( 'ninhursaga' );
\end{lstlisting}

The return value from a ``define'' of ``Yggdrasil::Entity'' is the
class the structure represents.  It will always be
``Namespace::Entityname'', and using the return value lets you rely on
Perls warnings and strict pragmas (assuming you use them) in case of a
typo.

\subsection{Creating properties}

Objects are created to store values, and to do this Yggdrasil requires
its user to define properties to its entities.  After we've defined
the property, it is instantly accessible for objects of the class in
question.  A property has a given type:

\begin{table}[h]
\centering
\label{types}

\begin{tabular}{l l}
Name & Definition  \\
\hline
TEXT & UTF-8 field of some size \\
BOOLEAN & A field which only accepts 0 or 1 as its values \\
VARCHAR(size) & A shorter text field, 1 to 255 units long \\
SERIAL & An automatically incrementing field \\
INTEGER & Signed, integers up to $2^16$ can be stored \\
FLOAT & Signed, floats of some size \\
DATE & A full date and time field, resolution varies \\
BINARY & A binary structure, up to 16MiB in size\\
IP & IPv4 or IPv6 in any form\\
\end{tabular}
\caption{Types in Yggdrasil}
\end{table}


\lstset{caption=Defining properties,label=properties}
\begin{lstlisting}
define Ygg::Host 'ip', 'Type' => 'IP';
define $hostclass 'comment';

# Set
$nommo->property( 'ip', '129.240.222.1' );
# Get
$nommo->property( 'ip' );
\end{lstlisting}

\subsection{Creating relations}

Relations are a fundamental piece of how Yggdrasil works.  Relations
create the edges in a network with entities as the nodes.  Yggdrasil
treats all edges as bidirectional and with identical weight.  

It is worth contemplating what kind of a network one is building.  In
a traditional case of three entities, ``Room'', ``Host'' and
``Person'', you can easily form a triangle.  Both a ``Person'' and a
``Host'' resides in a ``Room'', but a ``Host'' might also belong
directly to a ``Person'', not to a ``Room''.

This means we have a loop in our system, and if we ask Yggdrasil to
find what ``Host'' belongs to a specific person we can get two
distinct paths as the answer: Person->Room and Person->Host->Room.
Let us look at some code.

\lstset{caption=Defining relations,label=relations}
\begin{lstlisting}
# Create another entity, "Room".
my $roomclass = define Yggdrasil::Entity 'Room';
my $personclass = define Yggdrasil::Entity 'Person';
# Create a room object, 'b701' and a person object, 'terjekv'.
my $b701 = $roomclass->new( 'B701' );
my $terjekv = Ygg::Person->new( 'terjekv' );

# Now create the relation between rooms and hosts.
# $hostclass = 'Ygg::Host', $roomclass = 'Ygg::Room'
define Yggdrasil::Relation $hostclass, $roomclass;
\end{lstlisting}

This however doesn't do us much good, we need to link objects
together, and we do that as follows:

\lstset{caption=Linking ``nommo'' to ``b701'',label=linking}
\begin{lstlisting}
# Now, links the host 'nommo' to the room 'b701'
$nommo->link( $b701 );

# What rooms are nommo related to?
my @hits = $nommo->fetch_related( 'Room' );
# $hits[0] will be $b701.
\end{lstlisting}


\lstset{caption=Creating a loop,label=looping}
\begin{lstlisting}
# Let us create the loop
define Yggdrasil::Relation $hostclass, $personclass;
define Yggdrasil::Relation $room, $personclass;

$nommo->link( $terjekv );
$b701->link( $terjekv );

# Getting nommos room will again return B701.
@hits = $nommo->fetch_related( 'Room' );
\end{lstlisting}

Now, a small naughty...  Yggdrasil does \textit{not} require that all
answers of the same entities to be identical.  From the earlier
example, we now have two paths between Persons and Rooms: Person->Room
and Person->Host->Room.  Right now, those paths return the same
answer, but we're more than allowed to do something about that.

\lstset{caption=Two paths giving different answers,label=funpaths}
\begin{lstlisting}
my $b810 = $roomclass->new( 'B810' );

$b701->unlink( $terjekv );
$b810->link( $terjekv );
\end{lstlisting}

Oddly enough, we're now in the situation where two paths leading to
the same entity / class gives different objects.  Person->Room gives
us ``B810'' now, but Person->Host->Room goes terjekv->nommo->b701 and
thus gives us ``B701''.  Which one of these are correct?  Well, they
both are.  Which one you want can be controlled by specifying the
maximum distance allowed between the objects.   

A future addition to Yggdrasil might allow for constraints requiring
all paths leading between entities to give the same answer.  

\subsection{Object operations}

\newpage
\section{Internals}


\end{document}
