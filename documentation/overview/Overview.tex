\documentclass[english,a4paper]{article}
\usepackage[utf8]{inputenc}
\usepackage{babel,moreverb}
\usepackage[pdftex]{graphicx}
\usepackage{wrapfig}
\usepackage{float}
\usepackage{listings}
\usepackage{color}
 
\lstset{ %
language=Perl,                  % choose the language of the code
basicstyle=\small,              % the size of the fonts that are used for the code
numbers=none,                   % where to put the line-numbers
numberstyle=\footnotesize,      % the size of the fonts that are used for the line-numbers
stepnumber=2,                   % the step between two line-numbers. If it's 1 each line will be numbered
numbersep=5pt,                  % how far the line-numbers are from the code
backgroundcolor=\color{white},  % choose the background color. You must add \usepackage{color}
showspaces=false,               % show spaces adding particular underscores
showstringspaces=false,         % underline spaces within strings
showtabs=false,                 % show tabs within strings adding particular underscores
frame=single,                   % adds a frame around the code
tabsize=2,                      % sets default tabsize to 2 spaces
captionpos=b,                   % sets the caption-position to bottom
breaklines=true,                % sets automatic line breaking
breakatwhitespace=false,        % sets if automatic breaks should only happen at whitespace
escapeinside={\%*}{*)}          % if you want to add a comment within
                                % your code
}

\newcommand{\HRule}{\rule{\linewidth}{0.5mm}}

\begin{document}

\begin{titlepage}
 
\begin{center}
 
 
% Upper part of the page
%\includegraphics[width=0.15\textwidth]{./logo}\\[1cm]
 
% \textsc{\LARGE Yggdrasil documentation}\\[1.5cm]

\textsc{Yggdrasil documentation}\\[1.5cm]

 
\textsc{\Large Temporal storage of simple objects}\\[0.5cm]
 
 
% Title
\HRule \\[0.5cm]
{ \huge \bfseries Overview of Yggdrasil}\\[0.4cm]
 
\HRule \\[1.5cm]
 
%\begin{minipage}{0.4\textwidth}
%\begin{flushleft} \large
%\emph{Author:}\\
%Terje \textsc{Kvernes}
%\end{flushleft}
%\end{minipage}
%\begin{minipage}{0.4\textwidth}
%\begin{flushright} \large
%\emph{Author:} \\
%David \textsc{Ranvig}
%\end{flushright}
%\end{minipage}

Terje Kvernes \& David Ranvig

\vfill
 
% Bottom of the page
{\large \today}
 
\end{center}
 
\end{titlepage}


\tableofcontents

\newpage

\section{What is Yggdrasil?}

Yggdrasil aims to a ``dynamic relational temporal object database''.
In essence, Yggdrasil aims to add two abstractions to the traditional
view of a relational database: implicit temporal storage and a simple
object model to represent the data stored.  In addition to this
Yggdrasil allows the relations of the entities stored within to be
altered, and new entities and their relations to be added while the
system is running.  The relations are described by the administrator
of the system and as soon as any relation is described to Yggdrasil,
it is added to the overall structure of the installation.  

In Yggdrasil lingo you can think of an entity as a ``class'' from the
OO world.  The ``object'' is an instance of this entity, each object
existing in several temporal versions within the system.

Initializing Yggdrasil happens as follows, the parameters are all sent
to the back end storage layer, and both their meaning and their
necessity varies depending on said layer.  Look at the documentation
for Yggdrasil::Storage and its engines if in doubt.  The ``namespace''
parameter defines which namespace will house the entities we'll create
later.  This is, in essence, your class hierarchy.  You'll want to
ensure it's uncluttered.  

\lstset{caption=Initializing Yggrasil,label=ygginit,float=tp,aboveskip=0.7cm}
\begin{lstlisting}
new Yggdrasil(
              user      => user,
              password  => password,
              host      => host,
              db        => databasename,
              engine    => engine,
              namespace => 'Ygg',
             );
\end{lstlisting}

\subsection{Objects}

An object is an instance of an entity within the system.  This
instance is the primary working set that Yggdrasil operates on.
Objects contain properties which are key / value pairs.  Objects are
identified by a unique name within each entity.

%\begin{wrapfigure}{l}{5cm}
%  \begin{center}
%    \includegraphics[scale=0.1]{/usr/share/openclipart/png/computer/icons/lemon-theme/actions/messagebox_warning}
%  \end{center}
%  \small{You might see ``objects'' referred to as ``entities'' in Yggdrasil}
%\end{wrapfigure}

An object within Yggdrasil isn't a singular instance of grouped data.
As any change to this object is kept, every version of the object
throughout its existence is stored.  The default object is the
``current'' object, defined as the set of properties that are
currently active and not expired.

The limitations are currently bound to the objects being fairly
simple, they are not allowed to store anything more complex than
anything that can be mapped into a text field, and their only
relations to other objects are the ones defined by the setup of
Yggdrasil.  

To define an entity ``Host'' within Yggdrasil.  This will create
access to the class ``Ygg::Host'', as we earlier defined the namespace
for Yggdrasil to work in previously to be named ``Ygg''.  We'll then
create a ``Host'' object called ``nommo'' and one called
``ninhursaga''.

\lstset{caption=Defining entities,label=entity}
\begin{lstlisting}
$hostclass = define Yggdrasil::Entity 'Host';
my $nommo = $hostclass->new( 'nommo' );
my $ninhursaga = Ygg::Host->new( 'ninhursaga' );
\end{lstlisting}

The return value from a ``define'' of ``Yggdrasil::Entity'' is the
class the structure represents.  It will always be
``Namespace::Entityname'', and using the return value lets you rely on
Perls warnings and strict pragmas (assuming you use them) in case of a
typo.

\subsubsection{Properties}

Depending on the way Yggdrasil is set up you may or may note have
defined types for your properties, and you may or may not have
constraints to the data stored for each property.  

Yggdrasil can either be flexible and treat all property values as a
default type, or you can select between a set of types Yggdrasil
guarantees no matter the back end it's running on.

\lstset{caption=Defining properties,label=properties}
\begin{lstlisting}
define Yggdrasil::Entity 'ip', 'Type' => 'IP';
define $hostclass 'comment';
\end{lstlisting}

\subsection{Relations}

All relations within Yggdrasil exist between entities.  To create
relations between entities, define them, but we'll need another entity
to relate things to, since we have some hosts, let's make ourself some
rooms.

\lstset{caption=Defining relations,label=relations}
\begin{lstlisting}
# Create another entity, "Room".
my $roomclass = define Yggdrasil::Entity 'Room';
# Create a room object, 'b701'.
my $b701 = $roomclass->new( 'B701' );

# Now create the relation between rooms and hosts.
# $hostclass = 'Ygg::Host', $roomclass = 'Ygg::Room'
define Yggdrasil::Relation $hostclass, $roomclass; 
\end{lstlisting}

This however doesn't do us much good, we need to link objects
together, and we do that as follows:

\lstset{caption=Linking ``nommo'' to ``b701'',label=linking}
\begin{lstlisting}
# Now, links the host 'nommo' to the room 'b701'
$nommo->link( $b701 );
\end{lstlisting}
 
\subsection{Temporal dimension}

A principal idea of Yggdrasil is that data is only inserted, never
deleted.  This also means that there is a clear distinction between
``available'' and ``current'' data contained within the system.
As long as no time frame, or slice, is requested, all requests work on
the ``current'' dataset. 

Deletion only happens if and only if the object structure (``entity''
in Yggdrasil lingo) they belong to is deleted, and that deleted
structure is ``purged''.  Deletion of an entity is therefore, in some
circumstances, reversible.

Changes in the structure will retain the information if possible, the
system will inform you at any time if any action you take will
permanently delete any data.

\subsection{Dynamic storage}

New entities and new relations between entities can be issued on the
fly while the system is running live.  

% \subsection{Defining structure}

% Creating structure within Yggdrasil is done via commands prefixed
% ``define''.  This accounts for defining entities, properties and
% relations.  

% \begin{table}[h]
% \centering
% \caption{Define statements in Yggdrasil}

% \begin{tabular}{l l}
% \hline\hline %inserts double horizontal lines
% Statement & Parameters \\
% \hline
% define\_entity  & Name-of-entity \\
% define\_property & Name-of-entity, Name-of-property \\
% define\_relation & Name-of-entity1, Name-of-entity2 \\
% \end{tabular}

% \end{table}


\end{document}
