\documentclass[english,a4paper]{article}
\usepackage[utf8]{inputenc}
\usepackage{babel,moreverb}
\usepackage[pdftex]{graphicx}
\usepackage{wrapfig}
 
\newcommand{\HRule}{\rule{\linewidth}{0.5mm}}

\begin{document}

\begin{titlepage}
 
\begin{center}
 
 
% Upper part of the page
%\includegraphics[width=0.15\textwidth]{./logo}\\[1cm]
 
% \textsc{\LARGE Yggdrasil documentation}\\[1.5cm]

\textsc{Yggdrasil documentation}\\[1.5cm]

 
\textsc{\Large Temporal storage of simple objects}\\[0.5cm]
 
 
% Title
\HRule \\[0.5cm]
{ \huge \bfseries Overview of Yggdrasil}\\[0.4cm]
 
\HRule \\[1.5cm]
 
%\begin{minipage}{0.4\textwidth}
%\begin{flushleft} \large
%\emph{Author:}\\
%Terje \textsc{Kvernes}
%\end{flushleft}
%\end{minipage}
%\begin{minipage}{0.4\textwidth}
%\begin{flushright} \large
%\emph{Author:} \\
%David \textsc{Ranvig}
%\end{flushright}
%\end{minipage}

Terje Kvernes \& David Ranvig

\vfill
 
% Bottom of the page
{\large \today}
 
\end{center}
 
\end{titlepage}


\tableofcontents

\newpage

\section{What is Yggdrasil?}

Yggdrasil aims to a ``dynamic relational temporal object database''.
In essence, Yggdrasil aims to add two abstractions to the traditional
view of a relational database: implicit temporal storage and a simple
object model to represent the data stored.  In addition to this
Yggdrasil allows the relations of the entities stored within to be
altered, and new entities and their relations to be added while the
system is running.  The relations are described by the administrator
of the system and as soon as any relation is described to Yggdrasil,
it is added to the overall structure of the installation.  

In Yggdrasil lingo you can think of an entity as a ``class'' from the
OO world.  The ``object'' is an instance of this entity, each object
existing in several temporal versions within the system.

\subsection{Objects}

An object is an instance of an entity within the system.  This
instance is the primary data unit within Yggdrasil.

%\begin{wrapfigure}{l}{5cm}
%  \begin{center}
%    \includegraphics[scale=0.1]{/usr/share/openclipart/png/computer/icons/lemon-theme/actions/messagebox_warning}
%  \end{center}
%  \small{You might see ``objects'' referred to as ``entities'' in Yggdrasil}
%\end{wrapfigure}

An object within Yggdrasil isn't a singular instance of grouped data.
As any change to this object is kept, every version of the object
throughout its existence is stored.  

The limitations are currently bound to the objects being fairly
simple, they are not allowed to store anything more complex than
anything that can be mapped into a text field, and their only
relations to other objects are the ones defined by the setup of
Yggdrasil.

\subsection{Relations}

All relations within Yggdrasil exist between entities.  

\subsection{Temporal dimension}

A principal idea of Yggdrasil is that data is only inserted, never
deleted.  This also means that there is a clear distinction between
``available'' and ``current'' data contained within the system.
As long as no time frame, or slice, is requested, all requests work on
the ``current'' dataset. 

Deletion only happens if and only if the object structure (``entity''
in Yggdrasil lingo) they belong to is deleted, and that deleted
structure is ``purged''.  Deletion of an entity is therefore, in some
circumstances, reversible.

Changes in the structure will retain the information if possible, the
system will inform you at any time if any action you take will
permanently delete any data.

\subsection{Dynamic storage}

New entities and new relations between entities can be issued on the
fly while the system is running live.  

\end{document}
